\documentclass[twoside]{article}

\usepackage{lmodern}
\usepackage[T1]{fontenc}
\usepackage[spanish,activeacute]{babel}
\usepackage{mathtools}
\usepackage{graphicx}
\usepackage[inner=3.0cm,top=2.5cm,outer=2.0cm,bottom=2.5cm]{geometry}  
\usepackage{fancyhdr}
%\usepackage{hyperref}
%\usepackage[toc,xindy]{glossaries}

\pagenumbering{gobble}
%\renewcommand{\baselinestretch}{1.5}
\usepackage{setspace}
\onehalfspacing
\setlength{\footskip}{43.0pt}

\title{\begin{center} 
\includegraphics[width=0.8\textwidth]{images/upc-logo.png} 
\end{center} 
\vspace{0.5cm} 
Proyecto Final de Carrera\\
INGENIER\'IA INDUSTRIAL \\
\vspace{0.5cm} 
\Huge{
RosPiBot}
\vspace{0.5cm} \\
 Memoria}

\author{}
\date{} % void to avoid put current date
\pagestyle{fancy}

\fancyhead[LO,RE]{\title{RosPiBot}}
\fancyhead[LE,RO]{\thepage}
\fancyfoot[LE,RO]{\includegraphics[scale=0.5]{images/etseib-logo.png}}
\fancyfoot[C]{}
%\headrulewidth 0.4pt
%\footrulewidth 0 pt


\begin{document}
%\pagestyle{empty} 

\maketitle
\begin{center}
\large{$\begin{array}{ll}
\mbox{Autor:} & \mbox{Joan Guasch Iglesias} \\
\mbox{Director:} & \mbox{Manel Velasco Garcia} \\
\mbox{Convocatoria:} & \mbox{Fecha a presentar}
\end{array}$}\\ 
\vspace{2cm} 
\Large{Escuela T'ecnica Superior de Ingenier'ia Industrial de Barcelona}\\ 
\vspace{1cm}
\includegraphics[scale=1]{images/etseib-logo.png}
\end{center}
\thispagestyle{empty}
\newpage

\thispagestyle{empty}
\paragraph*{}
\newpage

%\thispagestyle{empty}
%\pagenumber{1}
\pagenumbering{arabic}
\fancyhead[LE,RO]{1} 

\begin{abstract}
Este proyecto consiste en el aprovechamiento de la plataforma rob'otica comercial WifiBot que corre el riesgo de quedarse obsoleta y rescatarla de su destino. Para ello se hace uso de una Raspberry Pi, un ordenador de placa reducida que ha revolucionado el mundo de la automatizaci'on desde el d'ia de su aparici'on en el 2012, como unidad de procesamiento de la nueva plataforma y que est'a basada en Linux. En el apartado de software, adem'as de incorporar herramientas que faciliten el trabajo a futuros usuarios se ha configurado para poder trabajar con ROS, una infrastructura digital para el desarrollo de software de robots creada por Willow Garage y extensamente utilizada.

A lo largo de este documento se expone el estado inicial del robot, los nuevos requerimientos a cumplir, las modificaciones efectuadas incluyendo las complicaciones encontradas a lo largo del proyecto. 
\end{abstract}
\thispagestyle{empty}
\newpage
\paragraph*{}
\thispagestyle{empty}
\newpage

\fancyhead[LE,RO]{3}

\tableofcontents
\setcounter{page}{1}
\addtocontents{toc}{~\hfill\textbf{P'agina}\par}
\addcontentsline{toc}{section}{Resumen}
%\setcounter{page}{3}
%\addcontentsline{toc}{section}{'Indice}
\newpage

%\pagenumbering{arabic}
\setcounter{page}{4}
\fancyhead[LE,RO]{\thepage}

\section{Glosario}
\newpage

%\section{Prefacio}
%\subsection{Motivaci'on}
%\newpage

\section{Introducci'on}
\subsection{Historia}
\subsection{Estado del arte}
\subsection{Objetivos}
\subsection{Alcance}


\newpage
\section{Material de partida }

\subsection{WifiBot}
\textbf{Wifibot} es una plataforma rob'otica, desarrollada por la empresa francesa \textbf{Nexter Robotics}, dise'nada para poder navegar en m'ultiples escenarios gr'acias a su dise'no. Su sistema de tracci'on a las cuatro ruedas, dise'no reducido y bajo peso, le otorga una gran flexibilidad.

\subsubsection{Descripci'on}
Este proyecto partir'a del modelo \textbf{WifiBot 4G}, producido durante el periodo 2002--2006 y que incluye las siguientes caracter'isticas:

\begin{figure}[ht]
\centering
\includegraphics[scale=0.5]{images/Visuel_Wifibot_2.png} 
\caption{Wifi Bot 4G}
\label{fig:Wifi Bot 4G}
\end{figure}

\paragraph{Componentes}
\begin{itemize}
\item CPU:
	\begin{itemize}
	\item Procesador AMD Au1500
	\item 400MHz
	\item Mem'oria RAM de 64MB
	\item Mem'oria Flash de 32MB
	\end{itemize}
\item Interf'icies:
	\begin{itemize}
	\item 4x Ethernet 10/100
	\item 1x USB
	\item 1x I$^{2}$C
	\item 1x RS232
	\end{itemize}
\item WIFI:
	\begin{itemize}
	\item WiFi con est'andar 802.11a/b/g
	\item Modos Access Point, Bridge, Client y Router
	\item 1x Antena de 5dBi
	\end{itemize}
\item Sensores:
	\begin{itemize}
	\item 1x C'amara IP
	\item 2x Sensor IR de dist'ancia
	\item 2x Codificador de efecto Hall (Hall Encoder) 
	\item 2x DSPIC30F2010
	\item 1x Nivel de bater'ia
	\end{itemize}
\item Motores:
	\begin{itemize}
	\item 4x Motor de 7.2V
	\item Reductora $i=50:1$
	\item Par nominal 8.87Kg/cm
	\item Velocidad nominal 120Rpm
	\end{itemize}
\item Dimensiones:
	\begin{itemize}
	\item Longitud 28cm
	\item Anchura 30cm
	\item Altura 20cm
	\item Peso 4.5Kg
	\end{itemize}
\item Bater'ias:
	\begin{itemize}
	\item 9.6V NiMh (8 celdas)
	\item Capacidad 9500mAh
	\item Autonom'ia de 2 horas
	\end{itemize}
\end{itemize}

\paragraph{Estructura}\noindent

La estructura de la base est'a formada por dos secciones sim'etricas que llamaremos hemisferios izquierdo y derecho. Estas dos partes est'an unidas por una barra roscada que atraviesa transversalmente todo el robot, ofreciendo un eje de rotaci'on entre los dos elementos, cualidad que le otorga una mayor adaptaci'on a superf'icies irregulares.
%include image SolidWorks
%como vemos en figura \ref{fig:Wifi Bot 4G}

\subsubsection{Estado Inicial}  
La plataforma de la que se dispon'ia en un principio carec'ia de cierta cualidades. La m'as destacable era la inestabilidad de la red WiFi, dicha conexi'on sufr'ia constantes ca'idas con lo que dificultaba enormemente su teleopraci'on. Su otro tal'on de Aquiles era su escasa documentaci'on disponible, al ser un producto que su fabricante da por descontinuado y al no disponer de una comunidad que lo mantenga, dicha base rob'otica provocaba quebraderos de cabeza para usuarios noveles. Finalmente, las bater'ias al haber completado su vida 'util, dispon'ian de una carga efectiva muy inferior a la inicial. 

En cambio, tanto la estructura, ruedas, motores y sensores se encontraban en mejores condiciones

\subsection{Raspberry Pi}
Raspberry Pi es un ordenador del tama'no de una tarjeta de cr'edito desarrolado en Reino Unido por la \textbf{Raspberry Pi Foundation}
\newpage

\section{Especificaciones}
Tal y como se ha definido anteriormente, el objetivo de este proyecto consiste en ofrecer una plataforma funcional para los futuros usuarios. Para definir esta ''funcionalidad'' se ha basado en el conocimiento aportado por antiguos usuarios, trabajadores en productos similares y en la experiencia propia. 

Las especificaciones se han clasificado seg'un su or'igen. Dependiendo de si son de car'acter \textbf{software}, \textbf{electr'onico} o \textbf{mec'anico}. Las especificaciones de car'acter inform'atico engloban aspectos como la interf'icie con el usuario, las herramientas de desarrollo (librer'ias, documentaci'on) y la disponibilidad a futuras modificaciones. Por contra, la electr'onica ha de evitar que el usuario se encuentre obligado a manipular el interior del robot, pero que en el caso de dicha situaci'on permita una soluci'on simple del problema. Adem'as, la electr'onica ha de incorporar un sistema que permita conocer estados del robot de manera sencilla. Finalmente, la mec'anica se encarga de incorporar las nuevas especificaciones en la plataforma, manteniendo el concepto inicial. Igual que en la electr'onica, en el caso de una futura manipulaci'on por parte del usuario, el dise'no ha de facilitar el acceso a cualquier ubicaci'on.

\subsection{Software}
El principal requisito del software es ofrecer un acceso simple al usuario. Teniendo presente el hecho que pueda ser utilizado tanto para usuario noveles como para usuarios experimentados. Por ello se ha planteado las siguientes especificaciones.

\subsubsection{Web}
Se ofrecer'a un servidor web que permita de visualizar f'acilmente el estado del robot como por ejemplo: el nivel de carga de bater'ias, sensores de proximidad, velocidad de desplazamiento, consumo de motores, etc. Adem'as se incluir'a una pantalla d'onde se monitorizar'a una c'amara que incorpora el robot y un par de terminales de comandos.
En otro apartado de la web se ofrecer'a la descarga de todos aquellos documentos necesarios para el usuario, evitando as'i que el usuario pierda tiempo en la b'usqueda de la documentaci'on.

\subsubsection{Librer'ias}
%Por facilidad de uso y comodidad se ha decidido que todo el software utilizado en este proyecto se base en $Python^{ TM}$, un lenguaje de programación
Para aquel desarrollador que decida utilizar esta plataforma se le ofrecer'a un conjunto de librer'ias que permitan trabajar con el robot de manera comoda. Estas librer'ias se compondr'an de una sintaxis limpia y un c'odigo legible que proporcinar'a una f'acil interpretaci'on.    

\subsection{Electr'onica}
El objetivo de la electr'onica es ser la parte m'as vital de la plataforma, pero pasando desapercibida por el usuario. Ofreciendo una m'inima interacci'on pero suficiente para comprender el estado del robot. Esta interacci'on se har'a mediante se'nales luminosas o ac'usticas, e indicar'an los siguientes estados.

\subsubsection{Funcionamiento}
En la plataforma de partida no se dispone de ning'un medio que indique si se encuentra operativo, por ello la primera especificaci'on de la parte electr'onica ser'a la incoporaci'on de un se'nal luminosa que muestre si el robot est'a encendido o apagado. Tambi'en existe la posibilidad de ofrecer informaci'on de otros estados.

\subsubsection{Alimentaci'on}
Las bater'ias ofrecer'an  una autonom'ia suficiente para moverse en un entorno exterior, se estipula un m'inimo de 2 horas en movimiento como m'inimo aceptable. Adem'as de la duraci'on, su recarga deber'a de ser lo m'aximo de comodo para el usuario, evitando que tenga que manipular las celdas de energ'ia. Se recomienda elementos de seguridad ante problemas como sobrecargas o cortocircuitos y de gesti'on de energ'ia para aumentar la vida 'util.

Otro punto a tener en cuenta es la disponibilidad en encontrar repuestos de la bater'ia, por ello se usar'an productos comerciales f'aciles de adquirir. Finalmente, se monitorizar'a el nivel de carga a partir de unos elementos luminosos.

\subsubsection{L'ogica}
Para el control de la plataforma el proyecto se ha basa en uso de la Raspberry Pi. Dicho ordenador ser'a el responsable de efectuar todo los c'alculos y 'ordenes, A'un as'i, a causa de sus limitaciones se precisa de una l'ogica complementaria capaz de gestionar y contabilizar las lecturas anal'ogicas de los sensores, el control de motores, la comunicaci'on externa y la distribuci'on de energ'ia.

\subsubsection{Comunicaci'on}  
Para poder interaccionar con el plataforma rob'otica se proponen 3 medios alternativos. Cada uno de estos canales permitir'an el acceso al usuario. Un primer medio ser'a por via cableada, a partir de un cable de red, los otros dos ser'an por via inal'ambrica, por un lado se buscar'a un red Wi-Fi predefinida y por otro se crear'a una red Wi-Fi propia. 

\subsection{Mec'anica}
Se intentar'a en la medida de lo posible mantener la silueta caracter'istica del robot, asignando mayor prioridad a aquellas modificaciones motivadas por software o  electr'onica. 

\subsubsection{Estructura}
Se desea mantener la estructura b'asica de la plataforma, dejando su peculiar forma de dos bloques unidos por una barra roscada longitudinal. Se mantendr'a el concepto de perfiles cuadrados como elemento estructural pero modificando la separaci'on entre ellos.La ubicaci'on de los distintos elementos se dispondr'an seg'un su relaci'on con el resto de dispositivos, uniendo seg'un se traten de control de potencia, l'ogica o comunicaci'on.


\subsubsection{Tornilleria}
A causa del deterioro de los elementos de tornilleria se proceder'a a cambiar el roscado de todos los tornillos por uno m'as estandarizado como el m'etrico. Este cambio tambi'en incluye las barras estructurales, los adaptadores de los ejes y los prisioneros



\newpage
\section{Modificaciones}
Para la realizaci'on de este proyecto se deber'a trabajar en varios sectores 


\subsection{Electr'onica}
\newpage

\section{Presupuesto}
\newpage

\section{Conclusiones}
\newpage

\section{Agradecimientos}
\newpage

\section{Biografia}
\newpage

\section{Soporte inform'atico}

\end{document}